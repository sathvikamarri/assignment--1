\documentclass[a4paper,12pt,two column]{article}
\usepackage[utf8]{inputenc}
\usepackage{amsmath}
\usepackage{amssymb}
\usepackage{multicol}
\usepackage{graphicx}
\begin{document}

{\title{Assignment I (ICSE-2019 CLASS 10)}}
{\author{Sathvika marri - AI21BTECH11020}}

\maketitle
\textbf{{ICSE-2019 CLASS 10}}\\\\
\textbf{Question: 6 (a)}\\\\
(a) In the given figure, \angle PQR = \angle PST = 90$^{\circ}$, PQ = 5 cm and PS = 2 cm.\\\\
(i) Prove that \triangle PQR \hspace{0.1cm} \sim \hspace{0.1cm}  \triangle PST.\\\\
(ii) Find \hspace{0.1cm} ratio \hspace{0.1cm} of \hspace{0.1cm} Area \hspace{0.1cm} of \triangle PQR \hspace{0.1cm} and\hspace{0.1cm}\\
Area \hspace{0.1cm}  of \hspace{0.1cm} quadrilateral \hspace{0.1cm} SRQT.

\begin{figure}[bht]
\includegraphics[width=5cm] {triangle PQR.png}
\caption{fig PQR}
\label{fig:fig PQR}
\end{figure}

\textbf{Solution:- }\\\\
(i)\hspace{0.1cm}  To \hspace{0.2cm} prove \hspace{0.2cm} \triangle PQR \hspace{0.1cm} \sim \hspace{0.1cm} \triangle PST\\\\
consider \hspace{0.1cm}  \triangle PQR \hspace{0.2cm} and \hspace{0.2cm} \triangle PST\\\\
\angle PQR = \angle PST \hspace{0.1cm} = 90$^{\circ}$ \hspace{0.1cm} (given)\\\\
\angle p \hspace{0.1cm} is \hspace{0.1cm} common.\\\\
\boxed{
\therefore \hspace{0.1cm} \triangle PQR \hspace{0.1cm} \sim \hspace{0.1cm} \triangle PST \hspace{0.2cm} (By \hspace{0.1cm} AAA \hspace{0.1cm} criterion)\\\\}

(ii) To find the ratio of area of \triangle PQR \hspace{0.1cm} and \hspace{0.1cm} area \hspace{0.1cm} of \hspace{0.1cm} quadrilateral \hspace{0.1cm} SQRT.\\\\
Now,
Area\hspace{0.2cm} of\hspace{0.2cm} \triangle PQR \hspace{0.2cm} is,\\\\ 
\implies
\[\frac{1}{2}\] \hspace{0.2cm} \times \hspace{0.2cm} PQ \hspace{0.2cm} \times \hspace{0.2cm} QR\\\\
from \hspace{0.2cm}the\hspace{0.2cm} given\hspace{0.2cm} diagram \hspace{0.2cm}we \hspace{0.2cm}can\hspace{0.2cm} say \hspace{0.2cm}\\
\frac{PQ}{PS} = \frac{QR}{ST}\\

PQ=5cm(given),QR=5cm(stated above)\\\\
\implies
\[\frac{1}{2}\] \times 5 \times 5 \hspace{0.1cm} = \hspace{0.1cm} \frac{25}{2}\\\\
Area \hspace{0.1cm} of\hspace{0.1cm} \triangle PST,\\\\
\implies
\[\frac{1}{2}\] \hspace{0.2cm} \times \hspace{0.2cm} PS \hspace{0.2cm} \times \hspace{0.2cm} ST\\\\
PS=2cm(given),ST=2cm(stated above)\\\\
\implies
\[\frac{1}{2}\] \times 2 \times 2 \hspace{0.1cm} = \hspace{0.1cm} \frac{4}{2}\\\\
Area \hspace{0.1cm}of\hspace{0.1cm} Quadrilateral\hspace{0.1cm} SQRT\hspace{0.1cm}=\hspace{0.1cm}Area\hspace{0.1cm} of\hspace{0.1cm} \triangle PQR - Area \hspace{0.1cm}of\hspace{0.1cm} \triangle PST,\\\\
\implies
\frac{25}{2}-\frac{4}{2} = \frac{21}{2}\\\\
Ratio,
\frac{25/2}{21/2} = \frac{25}{21}\\\\
\boxed{
\therefore ratio = \frac{25}{21}}
\end{document}

