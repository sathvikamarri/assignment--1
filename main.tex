\documentclass[a4paper,12pt,two column]{article}
\usepackage[utf8]{inputenc}
\usepackage{amsmath}
\usepackage{amssymb}
\usepackage{multicol}
\usepackage{graphicx}
\usepackage{setspace}
\usepackage{amsfonts}

\begin{document}
\title{Assignment 1\\\Large AI1110: Probability and Random Variables \\ICSE 2019 Grade 10}
{\author{Sathvika marri\\\normalsize AI21BTECH11020}}

% make the title area
\maketitle
\newpage
\bigskip

\textbf{Question: 6 (a)}

In the given figure, \angle PQR = \angle PST = 90$^{\circ}$,PQ = 5 cm and PS = 2 cm.

(i)Prove that \triangle PQR \sim  \triangle PST.

(ii)Find ratio of Area of \triangle PQR and Area of quadrilateral SRQT.

\begin{figure}[bht]
\includegraphics[width=8cm] {triangle PQR.png}
\caption{triangle PQR}
\label{fig}
\end{figure}

\textbf{Solution:- }

(i)To prove \triangle PQR \sim \triangle PST

consider  \triangle PQR and \triangle PST

\angle PQR = \angle PST  = 90 $^\circ$ (given)

\angle p is common.

\boxed{
\therefore \triangle PQR  \sim  \triangle PST (By AAA criterion)}

\vspace{1cm}

(ii) Now,Area of \triangle PQR  is,
\begin{align}
= \frac{1}{2} \times PQ  \times QR
\end{align}

From the given diagram we can say
\begin{align}
= \frac{PQ}{PS}= \frac{QR}{ST}
\end{align}

PQ=5cm(given),QR=5cm(stated above)
\begin{align}
= \frac{1}{2} \times 5 \times 5 = \frac{25}{2}
\end{align}

Area of triangle PST
\begin{align}
= \frac{1}{2} \times  PS \times ST
\end{align}

PS=2cm(given), ST=2cm(stated above)
\begin{align}
= \frac{1}{2} \times 2 \times 2 = \frac{4}{2}
\end{align}

Area of Quadrilateral SQRT = Area of triangle PQR - Area of triangle PST
\begin{align}
= \frac{25}{2}-\frac{4}{2} = \frac{21}{2}
\end{align}

Ratio, 
\begin{align}
\frac{25/2}{21/2} = \frac{25}{21}
\end{align}

\centering
\boxed{
\therefore ratio = \frac{25}{21}}

\end{document}
