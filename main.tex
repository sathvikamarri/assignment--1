\let\negmedspace\undefined
\let\negthickspace\undefined
%\RequirePackage{amsmath}
\documentclass[a4paper,12pt,two column]{article}
%
% \usepackage{setspace}
 \usepackage{gensymb}
%\doublespacing
 \usepackage{polynom}
%\singlespacing
%\usepackage{silence}
%Disable all warnings issued by latex starting with "You have..."
%\usepackage{graphicx}
\usepackage{amssymb}
%\usepackage{relsize}
\usepackage[cmex10]{amsmath}
%\usepackage{amsthm}
%\interdisplaylinepenalty=2500
%\savesymbol{iint}
%\usepackage{txfonts}
%\restoresymbol{TXF}{iint}
%\usepackage{wasysym}
\usepackage{amsthm}
%\usepackage{pifont}
%\usepackage{iithtlc}
% \usepackage{mathrsfs}
% \usepackage{txfonts}
 \usepackage{stfloats}
% \usepackage{steinmetz}
 \usepackage{bm}
% \usepackage{cite}
% \usepackage{cases}
% \usepackage{subfig}
%\usepackage{xtab}
\usepackage{longtable}
%\usepackage{multirow}
%\usepackage{algorithm}
%\usepackage{algpseudocode}
\usepackage{enumitem}
 \usepackage{mathtools}
 \usepackage{tikz}
% \usepackage{circuitikz}
% \usepackage{verbatim}
%\usepackage{tfrupee}
\usepackage[breaklinks=true]{hyperref}
%\usepackage{stmaryrd}
%\usepackage{tkz-euclide} % loads  TikZ and tkz-base
%\usetkzobj{all}
\usepackage{listings}
    \usepackage{color}                                            %%
    \usepackage{array}                                            %%
    \usepackage{longtable}                                        %%
    \usepackage{calc}                                             %%
    \usepackage{multirow}                                         %%
    \usepackage{hhline}                                           %%
    \usepackage{ifthen}                                           %%
  %optionally (for landscape tables embedded in another document): %%
    \usepackage{lscape}     
% \usepackage{multicol}
% \usepackage{chngcntr}
%\usepackage{enumerate}
\usepackage{tfrupee}
\usepackage{float}
\usepackage{algorithm2e}
%\usepackage{wasysym}
%\newcounter{MYtempeqncnt}
\DeclareMathOperator*{\Res}{Res}
\DeclareMathOperator*{\equals}{=}
%\renewcommand{\baselinestretch}{2}
%\renewcommand\thesection{\arabic{section}}
%\renewcommand\thesubsection{\thesection.\arabic{subsection}}
%\renewcommand\thesubsubsection{\thesubsection.\arabic{subsubsection}}

%\renewcommand\thesectiondis{\arabic{section}}
%\renewcommand\thesubsectiondis{\thesectiondis.\arabic{subsection}}
%\renewcommand\thesubsubsectiondis{\thesubsectiondis.\arabic{subsubsection}}

% correct bad hyphenation here
\hyphenation{op-tical net-works semi-conduc-tor}
\def\inputGnumericTable{}                                 %%

\lstset

\newtheorem{theorem}{Theorem}[section]
\newtheorem{problem}{Problem}
\newtheorem{proposition}{Proposition}[section]
\newtheorem{lemma}{Lemma}[section]
\newtheorem{corollary}[theorem]{Corollary}
\newtheorem{example}{Example}[section]
\newtheorem{definition}[problem]{Definition}
%\newtheorem{thm}{Theorem}[section] 
%\newtheorem{defn}[thm]{Definition}
%\newtheorem{algorithm}{Algorithm}[section]
%\newtheorem{cor}{Corollary}
\newcommand{\BEQA}{\begin{eqnarray}}
\newcommand{\EEQA}{\end{eqnarray}}
\newcommand{\define}{\stackrel{\triangle}{=}}
\newcommand*\circled[1]{\tikz[baseline=(char.base)]{
    \node[shape=circle,draw,inner sep=2pt] (char) {#1};}}
\bibliographystyle{IEEEtran}
%\bibliographystyle{ieeetr}
\providecommand{\mbf}{\mathbf}
\providecommand{\pr}[1]{\ensuremath{\Pr\left(#1\right)}}
\providecommand{\qfunc}[1]{\ensuremath{Q\left(#1\right)}}
\providecommand{\sbrak}[1]{\ensuremath{{}\left[#1\right]}}
\providecommand{\lsbrak}[1]{\ensuremath{{}\left[#1\right.}}
\providecommand{\rsbrak}[1]{\ensuremath{{}\left.#1\right]}}
\providecommand{\brak}[1]{\ensuremath{\left(#1\right)}}
\providecommand{\lbrak}[1]{\ensuremath{\left(#1\right.}}
\providecommand{\rbrak}[1]{\ensuremath{\left.#1\right)}}
\providecommand{\cbrak}[1]{\ensuremath{\left\{#1\right\}}}
\providecommand{\lcbrak}[1]{\ensuremath{\left\{#1\right.}}
\providecommand{\rcbrak}[1]{\ensuremath{\left.#1\right\}}}
\theoremstyle{remark}
\newtheorem{rem}{Remark}
\newcommand{\sgn}{\mathop{\mathrm{sgn}}}
\providecommand{\fourier}{\overset{\mathcal{F}}{ \rightleftharpoons}}
%\providecommand{\hilbert}{\overset{\mathcal{H}}{ \rightleftharpoons}}
\providecommand{\system}{\overset{\mathcal{H}}{ \longleftrightarrow}}
	%\newcommand{\solution}[2]{\textbf{Solution:}{#1}}
\newcommand{\solution}{\noindent \textbf{Solution: }}
\newcommand{\cosec}{\,\text{cosec}\,}
\providecommand{\dec}[2]{\ensuremath{\overset{#1}{\underset{#2}{\gtrless}}}}
\newcommand{\myvec}[1]{\ensuremath{\begin{pmatrix}#1\end{pmatrix}}}
\newcommand{\mydet}[1]{\ensuremath{\begin{vmatrix}#1\end{vmatrix}}}

\makeatletter
\@addtoreset{figure}{problem}
\makeatother
\let\StandardTheFigure\thefigure
\let\vec\mathbf

\title{
	%\logo{
%Computational Approach to School Geometry
	Assignment 1\\\Large AI1110: Probability and Random Variables - ICSE 2019 Grade10}
%	}

   
  \author{        Sathvika marri - ai21btech11020   % <-this % stops a space
                    }
\graphicspath{{figures/}}
%\title{
%	\logo{Matrix Analysis through Octave}{\begin{center}\includegraphics[scale=.24]{tlc}\end{center}}{}{HAMDSP}

\begin{document}

% make the title area
\maketitle
\newpage
\bigskip

\textbf{Question: 6 (a)}

In the given figure, \angle PQR = \angle PST = 90$^{\circ}$,PQ = 5 cm and PS = 2 cm.

(i)Prove that \triangle PQR \sim  \triangle PST.

(ii)Find ratio of Area of \triangle PQR and Area of quadrilateral SRQT.

\begin{figure}[bht]
\includegraphics[width=8cm] {triangle PQR.png}
\caption{triangle PQR}
\label{fig}
\end{figure}

\textbf{Solution:- }

(i)To prove \triangle PQR \sim \triangle PST

consider  \triangle PQR and \triangle PST

\angle PQR = \angle PST  = 90 $^\circ$ (given)

\angle p is common.

\boxed{
\therefore \triangle PQR  \sim  \triangle PST (By AAA criterion)}

\vspace{1cm}

(ii) Now,Area of \triangle PQR  is,
\begin{align}
= \frac{1}{2} \times PQ  \times QR
\end{align}

From the given diagram we can say
\begin{align}
= \frac{PQ}{PS}= \frac{QR}{ST}
\end{align}

PQ=5cm(given),QR=5cm(stated above)
\begin{align}
= \frac{1}{2} \times 5 \times 5 = \frac{25}{2}
\end{align}

Area of triangle PST
\begin{align}
= \frac{1}{2} \times  PS \times ST
\end{align}

PS=2cm(given), ST=2cm(stated above)
\begin{align}
= \frac{1}{2} \times 2 \times 2 = \frac{4}{2}
\end{align}

Area of Quadrilateral SQRT = Area of triangle PQR - Area of triangle PST
\begin{align}
= \frac{25}{2}-\frac{4}{2} = \frac{21}{2}
\end{align}
 
\begin{align}
\implies \frac{25/2}{21/2}=\frac{25}{21}
\end{align} 

\centering
\boxed{
\therefore ratio = \frac{25}{21}}

METHOD-2: steps for construction;

\begin{figure}[bht]
    \includegraphics[width=8cm]{Fig 2.jpg}
    \caption{triangle-2}
    \label{}
\end{figure}

From the above figure;

    $\frac{PQ}{PR}$= cos $\theta$

\begin{align}
    PR = \frac{PQ}{cos\theta}
\end{align}

    $\frac{QR}{PR}$= tan $\theta$

\begin{align}
    QR = tan \theta PQ
\end{align}

Given, PS=2cm PQ=5cm 

take an angle;
\angle RPQ = 60{\degree}

\begin{align}
    let \vec{Q} &=\myvec{0 \\ 0}\\
    P will be; \Vec{P} &=\myvec{-5 \\ 0}\\
    R will be; \Vec{R} &=\myvec{0 \\ -tan\theta PQ}\\
\end{align}

PT will be;\implies \frac{PS}{PT}=cos$\theta$
\begin{align}
    T will be; \Vec{T}=\myvec{-(5-\frac{PS}{cos\theta}) \\ 0}\\
\end{align}    
 
\begin{align}
    \frac{P^\prime S}{PS} = sin(90-\theta)
\end{align}     

\begin{align}
    P^\prime S = PS sin(90-\theta) 
\end{align}  
   
\begin{align}
    \frac{P^\prime P}{PS} = cos(90-\theta)
\end{align}   
   
\begin{align}
    P^\prime P = PS cos(90-\theta)
\end{align}    

\begin{align}
    \vec{S} &=\myvec{-(5 - PS cos\theta) \\ -Psin\theta}\\
\end{align}

Using the above coordinates,

Generating the figure using python:

\begin{figure}[bht]
    \centering
    \includegraphics[width=8.5cm]{constructed triangle.png}
    \caption{constructed triangle}
    \label{fig:my_label}
\end{figure}

The input and output parameters required for drawing the figure are available in the below table.\\
\begin{table}[!h]
    \begin{tabular}{|c|c|c|} \hline
        \textbf{Variable} & \textbf{Value}    & \textbf{Input/Output}          \\ \hline
        $R$               & 5               & Input          \\ \hline
        $\angle RPQ$    &$60^\circ$   &Input       \\ \hline    
        $\vec{Q}$       &\myvec{0\\0}        & Input  \\\hline
        $\vec{P}$       &\myvec{-5\\0}& Input\\\hline
        $\vec{R}$       &  $\myvec{0\\-5\tan60^\circ}$ & output\\\hline
        $\vec{T}$       &  $\myvec{-(5-2\sec60^\circ)\\0}$ & Output\\\hline
        $\vec{S}$       &  $\myvec{-(5-2\cos60^\circ)\\-5\sin60^\circ}$ & Output\\\hline
    \end{tabular}
\end{table}

\end{document}
